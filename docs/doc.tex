\documentclass[a4paper]{article}

\usepackage{amsmath}
\usepackage{graphicx}
\usepackage{graphics}
%\usepackage{enumerate}
\usepackage{natbib}
\usepackage{url} % not crucial - just used below for the URL 
\usepackage{amsfonts,amsthm,amsmath,amsthm}
\usepackage{hyperref,natbib}
\usepackage{setspace}
\usepackage{graphicx}
\usepackage{hyperref}
\usepackage{bm}
\usepackage{autobreak}
\usepackage[usenames,dvipsnames,svgnames,table]{xcolor}

\title
{DR kernel with Additive Tail}
\author{Nathan B. R. Wycoff}

\begin{document}
\maketitle

\section{Introduction}

In this repository we will study stationary kernels on $\mathbb{R}^P, P < \infty$, of the form, given a subspace $\mathcal{U} \subseteq \mathbb{R}^P$ of dimension $R$ and a basis for the rest of the space generating one dimensional subspaces $\mathcal{V}_i, i \in \{R+1, \ldots, P\}$:

\begin{equation}
	k(\mathbf{d}) = \gamma_0 e^{-\frac{1}{2}\mathbf{d}^\top \mathbf{P}_{\mathcal{U}} \mathbf{D}_l^{-1} \mathbf{P}_{\mathcal{U}}^\top\mathbf{d}} + \sum_{p=R+1}^P \gamma_{p}e^{-\frac{1}{2l_p}\mathbf{d}^\top \mathbf{P}_{\mathcal{V}_i}\mathbf{d}}
\end{equation}

here $\mathbf{d}$ denotes the displacement vector between two locations, $\mathbf{P}_{\mathcal{U}}$ gives the self-adjoint projection operator onto $\mathcal{U}$, and $\mathbf{D}_l$ is a diagonal matrix containing lengthscales.

Intuitively, the idea is that the kernel gives ``full attention" to the subspace $\mathcal{U}$ while still taking the rest of the space into account. 

Unfortunately, the basis chosen for the rest of the space is important, and will give different results. How to choose this basis is not yet clear to me, so I will do so randomly for now.

\section{Experiments}

\subsection{Low Rank Quadratic}

Here we will study the performance of the method with a fixed sample size on a quadratic function $f(\mathbf{x}) = \mathbf{x}^\top\mathbf{A}\mathbf{x}$ where $\textrm{rank}(\mathbf{A}) = R$.



\end{document}